\documentclass[11pt]{article}

\usepackage[margin=1.0in]{geometry}
\usepackage{url, enumitem}
\usepackage{amsfonts, amsmath, amsthm, amssymb}
\usepackage{listings}
\usepackage{hyperref}
\usepackage{color,soul}


\theoremstyle{definition}
\newtheorem{defn}{Definition}[section]
\theoremstyle{plain}
\usepackage[textsize=tiny]{todonotes}


\usepackage{proof}
\usepackage{bussproofs}

% Some useful macros.
\newcommand{\given}{\,|\,}
\newcommand{\R}{\mathbb{R}}
\newcommand{\C}{\mathbb{C}}
\newcommand{\E}{\mathbb{E}}
\newcommand{\var}{\text{var}}
\newcommand{\cov}{\text{cov}}
\newcommand{\p}{\partial}
\newcommand{\mba}{\mathbf{a}}
\newcommand{\mbb}{\mathbf{b}}
\newcommand{\mbx}{\mathbf{x}}
\newcommand{\mcX}{\mathcal{X}}
\newcommand{\mcY}{\mathcal{Y}}
\newcommand{\boldw}{\mathbf{w}}
\newcommand{\mbxt}{\tilde{\mathbf{x}}}
\newcommand{\Sigmat}{\tilde{\Sigma}}
\newcommand{\mbz}{\mathbf{z}}
\newcommand{\mbw}{\mathbf{w}}
\newcommand{\mcN}{\mathcal{N}}
\newcommand{\mcP}{\mathcal{P}}
\newcommand{\eps}{\epsilon}
\newcommand{\trans}{\intercal}
\newcommand{\Ut}{\tilde{U}}
\newcommand{\Beta}{\text{Beta}}
\newcommand{\Bernoulli}{\text{Bernoulli}}
\newcommand{\Elbo}{\text{ELBO}}
\newcommand{\KL}{\text{KL}}
\DeclareMathOperator*{\argmax}{arg\,max}
\DeclareMathOperator*{\argmin}{arg\,min}
\newcommand{\angstrom}{\textup{\AA}}
\renewcommand{\v}[1]{\mathbf{#1}}
\renewcommand{\b}[1]{\mathbb{#1}}

\hypersetup{
    colorlinks=true,
    linkcolor=blue,
    filecolor=magenta
    urlcolor=cyan,
}
\lstset{
    basicstyle=\ttfamily,
    mathescape
}


% Author: Mark Goldstein
% Date: Summer 2018
\begin{document}
\begin{center}
Semantics of Probabilistic Programs\\
Mark Goldstein
\end{center}


\section{Measures}

...

\section{Kernels} 

\noindent We need to briefly review kernels because
they show up in the semantics. A kernel $k$ from $X$ to $Y$, notated
$k: X \rightsquigarrow Y$, is a function $k: X \times \Sigma_Y \rightarrow [0,\infty]$
such that

\begin{itemize}
    \item for fixed $x$, $k(x,-): \Sigma_Y \rightarrow [0,\infty]$ is a measure
    \item for fixed $dy$, $k(-,dy): X \rightarrow [0,\infty]$ is a measurable function.
\end{itemize}

\noindent An intuitive example of kernels at work is conditional probability.
Take $k(x,-)$ to be a probability measure on $\Sigma_Y$ given a particular value $X=x$:
$\sum_{dy \in \Sigma_y} k(x,dy) = 1$. Note that $\sum_{x \in X} k(x,dy) \neq 1$ in general.\\

\noindent Let $X,Y,Z$ be measurable spaces and let
$k^1 : X \times Y \rightsquigarrow Z$ and 
$k^2 : X \rightsquigarrow Y$ be s-finite kernels (\textbf{TODO}: explain s-finiteness). 
Then we can define the composition $(k^1 \star k^2) : X \rightsquigarrow Z$ as

$$ (k^1 \star k^2)(x,U) = \int_Y k^2(x,dy) k^1(x,y,U) $$

\noindent For intuition, consider $k(x,-): \Sigma_Y \rightarrow [0,1]$
to be the probability measure that tells you how likely it is to start
off at location $x$ and end up in the interval $dy$. Then the composition
$(k^1 \star k^2)(x,dz) = \int_Y k^2(x,dy) k^1(x,y,dz)$ can be taken to represent
the probability of starting off at $x$ and ending up in the interval $dz$, where
we take a step in-between to land on $y$, but average out across all intervals
$dy$ that we can land in when jumping from $x$.\\

\section{Types and Semantics}

$$ \b{A},\b{B} ::== 
    \b{R} | P(\b{A}) | 1 | \b{A} \times \b{B} | \sum_i \b{A}_{i \in I} $$

\noindent where $I$ is countable and non-empty. Sum and product types. 
Reals. Distributions over $\b{A}$. $(1+1)$ is the type of bools, 
$P(1+1)$ is the type for distributions over bools, and $\sum_{i \in \b{N}} 1$ 
is the type for natural numbers.\\

\noindent \textbf{Typing judgements:} $\Gamma \vdash_d$ for deterministic
judgements and $\Gamma \vdash_p$ for probabilistic judgements. Having the two
typing judgements is mainly for notational clarity: it helps us to define 
interpretation differently for deterministic and probabilistic terms. \\

\noindent Types are interpreted as measurable spaces $[[\b{A}]]$. 


\subsection{Typing Judgements and Interpretations for Deterministic Terms}

\noindent Deterministic terms 
$\Gamma \vdash_d t : \b{A}$ are interpreted
as measurable functions $[[t]]: [[\Gamma]] \rightarrow [[\b{A}]]$.


\begin{itemize}
  \item $[[x]]_{\gamma, d, \gamma^\prime} = d$
  \item $[[(i,t)]]_\gamma = (i,[[t]]_\gamma)$
  \item $[[\texttt{case } t \texttt{ of } \{(i,x) 
         \rightarrow u_i\}_{i \in I}]]_\gamma = 
         [[u_i]]_{\gamma,d}$ if $[[t]]_\gamma = (i,d)$
  \item $[[()]]_\gamma = ()$
  \item $[[(t_0,t_1)]]_\gamma = ([[t_0]]_\gamma,[[t_1]]_\gamma)$
  \item $[[\pi_j(t)]]_\gamma = d_j$ if $[[t]]_\gamma = (d_0,d_1)$
\end{itemize}

\noindent where the $\texttt{case}$ expression above has a
deterministic continuation $u_i$. We will come back to the 
probabilistic continuation case. Now the semantics for sequencing.


\subsection{Typing Judgements and Interpretations for Probabilstic Terms}

\noindent Probabilistic terms $\Gamma \vdash_p t : \b{A}$ are interpreted
as s-finite kernels $[[t]]: [[\Gamma]] \rightsquigarrow [[\b{A}]]$.

\begin{itemize}

\item \noindent \textbf{\texttt{return}(t)}\\

\begin{center}
    \AxiomC{$\Gamma \vdash_d t : \b{A}$}
    \RightLabel{\quad \quad \quad $[[\texttt{return}(t)]](\gamma,da) =
      \delta_{[[t]](\gamma)}(da)$}
    \UnaryInfC{$ \Gamma \vdash_p \texttt{return}(t): \b{A}$}
    \DisplayProof
\end{center}


\noindent this corresponds to a Dirac Delta measure sitting at
a point $[[t]]_\gamma$. \\

\item \noindent \textbf{\texttt{let} x = t \texttt{ in } u}

\begin{center}
    \AxiomC{$\Gamma \vdash_p t : \b{A} \quad \Gamma,x:\b{A} \vdash_p u : \b{B}$}
    \RightLabel{\quad \quad \quad 
                $[[\texttt{let } x = t \texttt{ in } u]](\gamma,db) =
          \displaystyle{\int}_{x,dx \in [[\b{A}]]} 
                      [[u]] \Big( \gamma,x,db \Big) [[t]]\Big( \gamma,dx \Big)$}
    \UnaryInfC{$\Gamma \vdash_p \texttt{let } x=t \texttt{ in } u : \b{B}$}
    \DisplayProof
\end{center}

\noindent This one takes careful reading. Consider the kernel
composition definition above. Take $k_1$ to be $[[u]]$ and take
$k_2$ to be $[[t]]$. Then $(k^1 \star k^2)(\gamma, db) 
= ([[u]] \star [[t])(\gamma, db)$\\

\item \noindent \textbf{probabilstic \texttt{case}}

\begin{center}
  \AxiomC{$\Gamma \vdash_d t : \sum_{i \in I} \b{A}_i$}
  \AxiomC{$(\Gamma,x:\b{A}_i \vdash_p u_i : \b{B})_{i \in I}$}
  \BinaryInfC{$\Gamma \vdash_p (\texttt{case } t \texttt{ of }
              \{(i,x) \rightarrow u_i \}_{i \in I}): \v{B}$} 
  \DisplayProof
\end{center}

$$ [[\texttt{case } t \texttt{ of } \{(i,x) \rightarrow u_i \}_{i \in I} ]]
    (\gamma,db) = [[u_i]](\gamma::d,db) \texttt { if } [[t]](\gamma) =(i,d)$$

\item \noindent \textbf{\texttt{sample}(t)}

\begin{center}
    \AxiomC{$\Gamma \vdash_d t : P(\b{A})$}
    \RightLabel{\quad \quad \quad 
      $[[\texttt{sample}(t)]](\gamma,da) = 
      \Big([[t]](\gamma)\Big)(da)$}
    \UnaryInfC{$ \Gamma \vdash_p \texttt{sample}(t): \b{A}$}
    \DisplayProof
\end{center} 

\item \noindent \textbf{\texttt{score}(t)}

\begin{center}
    \AxiomC{$ \Gamma \vdash_d t : \b{R}$}
    \UnaryInfC{$\Gamma \vdash_p \texttt{score}(t): 1$}
    \DisplayProof
\end{center}

\[ 
  [[\texttt{score}(t)]](\gamma,du) = 
    \left\{\begin{array}{lr}
     abs \Big( [[t]](\gamma) \Big), & \text{if } du=\{()\}\\
     0, & \text{if } du=\emptyset 
    \end{array}\right\}
\]

\end{itemize}

\subsection{Typing Judgements and Interpretation for Normalize}

\begin{center}
    \AxiomC{$\Gamma \vdash_p t : \b{A}$}
    \UnaryInfC{$\Gamma \vdash_d \texttt{normalize}(t): \b{R} \times P(\b{A}) + 1 + 1$}
    \DisplayProof
\end{center}

\noindent To give it a semantics, we must find the normalizing constant to 
divide by. Consider $\Gamma \vdash_p t : \b{A}$ and 
let $\texttt{evidence}_t = [[t]]_{\gamma,[[\b{A}]]}$. Then:

\[ 
  [[\texttt{normalize}(t)]]_{\gamma} = 
    \left\{\begin{array}{lr}
    (0,(\texttt{evidence}_t, \frac{[[t]]_{\gamma,(-)}}{\texttt{evidence}_t})), &
            \text{if } \texttt{evidence}_t \in (0,\infty)\\
    (1,()), & \text{if } \texttt{evidence}_t = 0\\
    (2,()), & \text{if } \texttt{evidence}_t = \infty
    \end{array}\right\}
\]

\section{Example Programs}

...

\section{Commutativity}

...

\section{Higher Order}

...

% \infer{B}{A & (A \rightarrow B)}  


\end{document}



